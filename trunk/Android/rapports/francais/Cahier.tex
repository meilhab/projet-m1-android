\chapter{Cahier des charges}

\section{Conception de l'application}

On peut d�composer l'application en diff�rentes grandes parties.\\
Une partie g�n�rale qui reprend les objectifs du logiciel de type RCP.\\
Une partie concernant le stockage des informations sur le smartphone.\\
Une derni�re partie permettant de g�rer le serveur distant.

\noindent Les objectifs g�n�raux de l'application sont :
\begin{itemize}
    \item une interface graphique fluide et agr�able d'utilisation tout en restant compl�te et performante;
    \item la gestion des tags (ajout et suppression);
    \item la gestion des t�ches (ajout, modification et suppression);
    \item la gestion des t�ches filles avec les t�ches m�re;
    \item les diff�rents tris possibles des t�ches;
    \item l'internationalisation de l'application en anglais;
    \item la gestion des pr�f�rences de l'application.

\end{itemize}

\vspace{0.25cm}

\noindent Concernant la base de donn�e :
\begin{itemize}
    \item la cr�ation d'une base coh�rente permettant de g�rer les diff�rentes donn�es de l'application;
    \item le stockage, la modification et la suppression de donn�es dans l'application.

\end{itemize}

\vspace{0.25cm}

\noindent Concernant la synchronisation avec un serveur distant :
\begin{itemize}
    \item la cr�ation d'une base plus �volu�e que celle du smartphone;
    \item l'envoi et la r�ception des donn�es avec leur stockage, modification et suppression;
    \item la gestion des utilisateurs;
    \item la mise en place de diff�rentes m�thodes de synchronisation :
    \begin{itemize}
        \renewcommand{\labelitemii}{$\rightarrow$}
        \item �crasement des donn�es du smartphone et remplacement par celles du serveur;
        \item �crasement des donn�es du serveur et remplacement par celles du smartphone;
        \item combiner les donn�es du serveur et du smartphone.

    \end{itemize}
    \item la gestion d'un serveur proxy.

\end{itemize}


\section{Contraintes techniques}

Le d�veloppement d'une application Android est soumis � quelques contraintes pour arriver � un r�sultat.

\subsection{Les outils de d�veloppement}

Pour pouvoir facilement d�velopper avec Android, il a fallu installer 

\subsection{Le smartphone de test}

\subsection{Le serveur web}

\subsection{Divers}


\section{Contraintes temporelles}



\clearpage
